% article example for classicthesis.sty
\documentclass[a4paper,dottedtoc,headinclude,footinclude]{report} % KOMA-Script article scrartcl

\usepackage[pdfspacing,beramono,eulermath]{classicthesis} % nochapters
\usepackage{arsclassica}
\usepackage{lipsum}
\usepackage{url}
\usepackage{graphicx}
\usepackage{float}
\usepackage{amsthm, latexsym}
\usepackage{amssymb}
\usepackage{bm}
\usepackage{caption}
\usepackage{marginnote}
\usepackage{bbold}
\usepackage{minted}
\usepackage[T1]{fontenc} % T2A for cyrillics
\usepackage[utf8]{inputenc}				

\newcommand{\R}{{\mathbb R}}
\newcommand{\Q}{{\mathbb Q}}
\newcommand{\C}{{\mathbb C}}
\newcommand{\N}{{\mathbb N}}
\newcommand{\Z}{{\mathbb Z}}

\theoremstyle{plain}
%\numberwithin{equation}{section}
\newtheorem{thm}{Theorem}[section]
\newtheorem{theorem}[thm]{Theorem}
\newtheorem{lemma}[thm]{Lemma}
\newtheorem{example}[thm]{Example}
\newtheorem{definition}[thm]{Definition}
\newtheorem{proposition}[thm]{Proposition}

\begin{document}

    \title{\rmfamily\normalfont\spacedallcaps{Scientific Writing -- $8641$}}
    \author{\spacedlowsmallcaps{Massimo Nocentini}}
    \date{\today} 
    
    \maketitle
    
    \begin{abstract}
        \noindent In this report we collect our work during classes of \emph{Scientific Writing -- 8641}
        course, taught in Florence during Winter 2016.
    \end{abstract}
       
    \tableofcontents
    

    \chapter{Sentence structure}
    
    \section{Homework}

    Alan Turing was a computer science, a mathematician and a logician,
    born in England in June 1912. Most people know him for his
    fundamental work in breaking the Nazi's Enigma code during World War
    II. Although these were impressive results, Turing's major interest was in
    the basis on which math lies.  In order to formally state the notion
    of ``computation'', he defined his ``machines'', an abstract model of
    a modern computer. Those machines allowed him to prove that there
    is a machine, called Universal, which is able to simulate the behavior of
    any other machine; furthermore, Turing proved that there is a
    problem, the most general one is the Halting problem, which any 
    given machine cannot solve. 
    In spite of his genius, the British government condemned him for
    homosexual acts. Chemical treatments brought on his premature death;
    he committed suicide at the age of 41.

    \chapter{Paragraph structure}
    
    \section{Homework}

    Nonetheless oranges and lemons are great sources of vitamin C,
    kiwifruit is packed with more vitamins, copper and fiber;
    moreover, kiwifruit has lots of health benefits and can be mixed in with
    many other ingredients. While fresh citrus are available during
    the Winter, kiwifruit is exported from California, New Zealand and
    Italy all over the year. From the nutritional intake point of view,
    70 grams of kiwifruit are equivalent to 45 calories,
    providing vitamins C, K and E, dietary fiber and potassium: this
    mixture reduces high cholesterol levels and may reduce the risk of
    heart diseases. In particular, it could be dispensed to diabetic
    patients to keep their blood sugar level under control. Its
    unique flavor can be tasted as is, mixed in with tossed green salads
    or topped with yogurt together with strawberries.  Kiwifruit
    should be included in children diets and in adults as well;
    however its small dimension, it is a good food for everyone.
    
    \chapter{Unity, coherence and cohesion}
    
    \section{Homework}

    Jigoro Kano, was a japanese educator and martial artist, born 
    in December of 1860, is recognized as the founder of Judo,
    namely a collection of techniques and principles taken from
    ancient martial artists and adapted to be a way of life.
    Kano believed that body, mind and soul should be trained as
    a whole and not as individual entities; for this reason, 
    he coined a personal martial art, based on the concept of 
    maximum efficiency with minimum effort, 
    it differs from existing ones in the cohesion of its mixture: 
    he took, first, grappling and throwing techniques from jujitsu;
    second, principles and formality from karate and,
    finally, mindfulness breathing and zen practice
    from japanese masters. In parallel to his athletic life, Kano
    taught economics at the University of Tokio and later was elected 
    as Ministry of Education; in both tasks, Kano tried to provide
    a new way of teaching, based on mutual assistance and on
    understanding the importance of individual contribution to 
    build a better society, rather than achieving goals, personally.
    In Kano's words: "Of course, I am not negating the importance of 
    wanting to become strong or skilled. However, it must be 
    remembered that this is just part of the process for a greater 
    objective...The worth of all people is dependent on how they spend 
    their life making contributions."
    Kano coded its methodology in a set of "kata", stylized movements
    catching the essence of his thoughts: such actions cannot be
    applied directly in a fight, but show sequences of steps that are necessary
    to properly perform every other techniques.
    Nowadays Judo is an Olimpic sport and many judokas practice it at all
    competitive levels; however, the traditional way isn't lost and
    many masters taught it strictly according to original Kano's style.

    


    \iffalse
    % bib stuff
    \nocite{*}
    \addtocontents{toc}{\protect\vspace{\beforebibskip}}
    \addcontentsline{toc}{section}{\refname}    
    \bibliographystyle{plain}
    \bibliography{../Bibliography}
    \fi

\end{document}
