
\chapter{Linkers}

\section{Provided links}

\begin{itemize}
    \item \url{http://www.elc.byu.edu/classes/buck/w_garden/classes/buck/transitions.html}
    \item \url{http://grammar.ccc.commnet.edu/grammar/transitions.htm}
\end{itemize}

\section{Homework}

\subsection{Narrating Your Professional Life: Writing the Academic Bio}

by Jennifer Sano-Franchini\footnote{\url{http://www.gradhacker.org/2011/09/23/narrating-your-professional-life-writing-the-academic-bio/}} on September 23, 2011\\\\
For our launch in June, I wrote a GradHacker post on Writing the Academic Conference Proposal. Since one commenter made the fantastic suggestion that we follow up with a post on writing an academic bio, I decided to do just that! This post is based on my limited experience writing and reading a variety of academic bios, mostly in the humanities, and in rhetoric and composition more specifically. I also think several of these suggestions are up for debate and may vary depending on things like disciplinary conventions or personal preference, so these tips are by no means intended to be prescriptive.



Like Maureen suggests, writing an academic bio is hard. It's a unique kind of writing that can be especially difficult for those who are new to it. As she expresses, it might feel like you haven't accomplished much to write about as a lowly graduate student. Or it might seem daunting to think this thing that is supposed to represent you is going to be accessible to a wide audience in print or on a website where you can't just up and change it anytime you want. It might be difficult to write about your accomplishments in a way that feels self-congratulatory. Or it might also be difficult to write about yourself in such a compartmentalized way–it might feel like your bio doesn't really say all that much about you as a whole person with many facets. I think understanding the academic bio as a writing genre that performs specific writerly moves helps. It may also help to keep in mind that because academic bios are generally written for traditional, institutional spaces and situations, they are oftentimes quite conventional across the board. Here are some tips for writing an academic bio:

First, three big picture things to keep in mind that will pretty much always outweigh any smaller, more specific tips: context, audience, and purpose.

For instance, some of the common contexts for academic bios include: publications (traditional print \& digital, open access), conference proposals and proceedings, fellowship or other types of funding applications, course websites, professional websites and blogs, departmental/institutional websites, and social/professional networking sites like Twitter. Context also includes things like disciplinary conventions, so it's a good idea to look at some bios in your discipline.

Some of the common audiences who read academic bios include: colleagues/academics in your department or discipline (including people who could be on one of your hiring committees in the future!), academics outside of your field of study, undergraduate students, and clients of various types. It's definitely important to keep in mind that any and all of these audiences could potentially encounter your bio with a quick Google search, but it's nonetheless a good idea to tailor your bio for your specific target audience(s).

Some of the common purposes of academic bios include: to give readers of an article or conference proceeding a sense of who is providing that information; to acquaint another academic interested in your research with some of your background information; to give clients of a particular institutional site a sense of who they're working with; to give prospective graduate students a sense of the grad students who are currently in a department; to give undergraduate students a sense of who their instructors are; to contribute to an institutional, departmental, or programmatic identity; or to give potential collaborators or potential hirers a sense of the work you do along with your academic and scholarly identity.

Context, audience, and purpose matter because they should help you decide what information about yourself you'll want to emphasize. With these three points in mind, you'll want to think about things like: What kinds of information will my audience be looking for in this particular context? What kinds of information will my audience be interested in for this particular context? Is my tone/style appropriate for this context/audience/purpose?

\subsubsection{What to include}
What you include in your bio will depend on the aforementioned factors, but these factors will often be presented in the guise of length restrictions. Below are some conventional tips for traditional academic bios.

The Barebones Bio. For a very short bio (35-50 words), generally used in things like publications and conference proceedings, you should include the basics:

your name,
position,
department,
institution, and
research interests.
You might present your research interests using a short list, or with a sentence-length description of your dissertation, thesis, or other major project. If you're writing your bio for something that has a pedagogical focus, you might include your teaching interests and experiences instead of or alongside your research interests.

Here's a fill-in-the-blank example of the barebones bio:

\begin{verbatim}
[[Name]] is a [[Position]] in the [[Department]] 
(with a specialization/concentration in [[Specialization]]) 
at [[Institution]]. S/he is interested in [[research/teaching interests]]. 
(Or, her/his research/pedagogical interests include...) More specifically, 
her/his work examines (or other such verb) [[fill in the blank]].
\end{verbatim}

The Mid-Length Bio. For a mid-length bio (100-200 words), for something like an institutional or departmental website, you might add:

degrees held,
recent or ongoing scholarly projects,
notable awards and honors,
publications,
journals in which you've published, or
you might simply situate your research interests in a larger field of study.

The Longer Bio. For a longer bio (200-400 words), for something like your professional website, you might add:

non-academic interests and hobbies, and/or
information about your background (especially if it is somehow relevant to your research interests).

\subsubsection{Organizing your bio}

Here are a few different ways people organize academic bios:

Present-Focused

Broad to narrow (i.e., general research interests to more specific projects)
Narrow to broad (i.e., specific object of analysis that is then situated in a larger disciplinary conversation)
Timeline/Trajectorial (Generally for a mid-length to longer bio)

Makes connections between projects–for example, work done in the past, work being done in the present, and where that work is going in the future. The components might be arranged in different ways, for example, beginning with a present project, rooting that project in past experiences, and describing future paths. Or there may be just a couple of these components, for example, present and past work, or present and future work.
Thematic

Alternately, scholarly projects and activities may be organized according to theme (i.e., topic, theory, or methodology).

\subsubsection{Voice}

First or Third Person?

Another decision you may need to make is whether you will use the first or third person. Sometimes, depending on the context for your bio, this information is provided for you. If not, you may want to check what others are doing in the same journal/website/other. If you see that there doesn't seem to be a consistent guideline, or if the bio is for a site over which you have control, you'll need to make a decision: How does the voice fit the context/audience/purpose? How does the voice fit how you as an academic want to be perceived?

What other tips do you have for writing an academic bio? How are the conventions for writing an academic bio different in your discipline? What concerns and factors have I missed?

\subsection{My own biography}

Take a look at \url{http://templatelab.com/biography-templates/} to get ideas.
