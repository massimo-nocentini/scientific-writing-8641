
\chapter{Paragraph structure}

\section{Provided links}

\begin{itemize}
    \item \url{https://www.dlsweb.rmit.edu.au/lsu/content/4_writingskills/writing_tuts/paragraphs_ll/model.html}
    \item \url{https://owl.english.purdue.edu/engagement/2/1/29/}
    \item \url{http://www.uefap.com/writing/exercise/parag/paragex6.htm}
    \item \url{http://www.studyzone.org/testprep/ela4/j/supportsentp.cfm}
    \item \url{http://www.uefap.com/writing/exercise/parag/paragex5.htm}
    \item \url{http://writesite.elearn.usyd.edu.au/m3/m3u2/m3u2s5/m3u2s5_1.htm}
    \item \url{http://grammar.ccc.commnet.edu/grammar/composition/organization.htm}
\end{itemize}

\section{Homework}

Despite the fact that oranges and lemons are great sources of vitamin C,
kiwifruit is packed with more vitamins, copper and fiber;
moreover, kiwifruit has lots of health benefits and can be combined in with
many other ingredients. While fresh citrus is available during
the winter, kiwifruit is exported all year long from California, New Zealand and
Italy. From the nutritional intake point of view,
70 grams of kiwifruit are equivalent to 45 calories,
providing vitamins C, K and E, dietary fiber and potassium: this
mixture reduces high cholesterol levels and may reduce the risk of
heart disease. In particular, it can be dispensed and administered
for drugs to diabetic
patients to keep their blood sugar level under control. Its
unique flavor can be tasted as is, mixed in with tossed green salads
or topped with yogurt together with strawberries.  Kiwifruit
should be included in both children's and adults' diets;
despite its small dimension, it is a good food for everyone.
