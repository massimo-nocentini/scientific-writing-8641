
\chapter{Sentence structure}

\section{Homework}

Alan Turing was a computer science, a mathematician and a logician,
born in England in June 1912. Most people know him for his
fundamental work in breaking the Nazi's Enigma code during World War
II. Although these were impressive results, Turing's major interest was in
the basis on which math lies.  In order to formally state the notion
of ``computation'', he defined his ``machines'', an abstract model of
a modern computer. Those machines allowed him to prove that there
is a machine, called Universal, which is able to simulate the behavior of
any other machine; furthermore, Turing proved that there is a
problem, the most general one is the Halting problem, which any 
given machine cannot solve. 
In spite of his genius, the British government condemned him for
homosexual acts. Chemical treatments brought on his premature death;
he committed suicide at the age of 41.
