

\chapter{From paragraph to paper}

\section{Homework}

The following is an outline about \emph{handwritten letters} topic; ideas are taken from 
\footnote{\url{https://www.reviveourhearts.com/true-woman/blog/10-reasons-to-reclaim-the-lost-art-of-letter-writi/}}
and \footnote{\url{http://www.theguardian.com/culture/charlottehigginsblog/2012/oct/23/lost-art-letter-writing}}.

\begin{itemize}
    \item handwritten letters should be revived for the amount of soul, sentiments and
            long lasting presence they preserve

    \begin{itemize}
        
        \item bring writer's personal style

            \begin{itemize}
                \item see, touch and follow writer's handwriting flow
                \item how the writer accentuate letters and put emphasis on them, showing its 
                        character at the time the letter is written
                \item even before reading words, the writer has sent something nobody else 
                        could have sent
            \end{itemize}

        \item letters takes relationships on

            \begin{itemize}
                \item impress sentiments that the recipient could reread
                \item letters are deliberate acts of exposure, a form of vulnerability
                \item outliving both the writer and the receiver, letter could
                        be opened by loved ones keeping relations in the present
            \end{itemize}

        \item letters keeps time and effort

            \begin{itemize}
                \item requires time slots to prepare, think, write and send: this
                        communicates to the recipient how it is important
                \item allows the writer to speak slowly, letters are forms of "slow communications"
                \item gives to the recipient the gift of time without the pressure to reply
                        as soon as possible, even to not reply at all
            \end{itemize}

    \end{itemize}

    \item letters can console the grieving, strengthen the weary, and soften hearts
\end{itemize}

Here are my introduction and first body paragraphs:
\\\\
In our busy, over-connected and realtime daily routine we found handwriting a "strange"
thing to do: why do we have to put thoughts on a piece of paper written by hand, when the same
message can be written on our laptops typing on keyboards? Should it be faster,
chiper and instantaneous, doesn't it? Sure, it does. But our digital devices cannot
reproduce our styles \emph{doing} such writing: hands driven by our feelings; our handwriting
flows; our way to leave ink on a piece of paper. Moreover, digital text messages, emails, tweets and
blog posts live as long as we let them to; on the other hand, handwritten manuscripts 
outlive us and can be kept safely, accumulating sentiments in the meanwhile.
Handwritten letters should be revived for the amount of soul, sentiments and 
long lasting presence they preserve.

As soon as the writer starts writing a letter by hand, it opens itself up: it shows
its calligraphy, how it trace glyphs with the pensil, how much pressure it does on the paper.
The recipient can, slowly and gently, follow ink tracks with its fingers, trying 
to understand the soul the writer was experiencing at the time it was composing the letter. 
Writer's character is encoded
in each letter, willing or not, making compositions unique; even before recipient
starts reading, the writer has sent something that nobody else could have sent. 
This is an opportunity -- and a way -- for the writer to show respect, 
importance and value to its recipient;
at the same time, receiving an handwritten letter is for the recipient a soft request of 
reciprocal involvement, to take care of relationship at the same level the writer does.
