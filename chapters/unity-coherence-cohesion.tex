
\chapter{Unity, coherence and cohesion}

\section{Homework}

Jigoro Kano, a Japanese educator and martial artist born in
December 1860, is recognized as the founder of Judo, namely a
collection of techniques and principles taken from ancient martial
artists, combined together to be a way of life.  Kano believed that body,
mind and soul should be trained as a whole and not as individual
entities; for this reason, he coined Judo as a personal martial
art, based on the concept of maximum efficiency with minimum
effort.  Judo differs from existing disciplines in the cohesion of
its mixture: Kano took, first, grappling and throwing techniques
from jujitsu; second, principles and rigor from karate and,
finally, mindful breathing and zen practice from Japanese
monks. Along with to his athletic life, Kano taught economics at
the University of Tokyo and later was elected as Ministry of
Education; in both tasks, Kano tried to provide a new way of
teaching, based on mutual support and on understanding the
importance of individual contribution to build a better society,
rather than achieving personal goals. In Kano's words: "Of
course, I am not negating the importance of wanting to become
strong or skilled. However, it must be remembered that this is
just part of the process for a greater objective...The worth of
all people is dependent on how they spend their life making
contributions."  Kano coded its martial style in seven "kata"s,
which are sets of stylized movements, aiming to capture the essence
of his thoughts: such actions cannot be applied directly in a
fight, but show sequences of steps that are necessary to properly
perform every other Judo technique.  Nowadays Judo is an Olympic
sport and many judokas practice it at all competitive levels;
however, the traditional way isn't lost and many masters teach it
strictly according to original Kano's style.
